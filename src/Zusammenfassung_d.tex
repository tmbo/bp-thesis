%!TEX root = ../Bachelorarbeit.tex
\addchap*{Zusammenfassung}

Im Verlauf der kreativen Design Prozesse in der HPI School of Design Thinking (D-School) entstehen eine Menge an Dokumentationsartefakten. Diese Daten sind jedoch meist in Quellen abgelegt, deren gesicherte Persistenz nur für einen beschränkten Zeitraum gegeben ist. Die Informationen aus den Wissensquellen gilt es dauerhaft zu sichern,  organisieren und für Außenstehende verständlich darzustellen. Durch die graphische Darstellung des Prozesses der Ideenfindung können sowohl die Teilnehmer eines Design Thinking-Teams selbst, als auch deren Mentoren wichtige Erkenntnisse für den weiteren Erfolg einer Idee gewinnen. Neben diesem Überblick über die Entwicklung in einem Projekt, fehlt der D-School eine Gesamtübersicht über all ihre Projekte. Diese Hilft, Abhängigkeiten der Projekte voneinander zu verdeutlichen und erleichtert die Akquirierung neuer Projekt-Themen. 

In dieser Bachelorarbeit soll das Backend der entwickelten Lösung \tete{Project-Zoom} näher erläutert werden. Die Architektur dieses Systems wird betrachtet und die zugrundeliegenden Entscheidungen erklärt. Dazu werden zu Beginn die Anforderungen des Projektpartners aufgezeigt und im Verlauf der Arbeit als Entscheidungsgrundlage verwendet. Das Backend der umgesetzten Anwendung basiert auf einer asynchronen Webarchitektur. Es wird das Konzept des Data-centric Designs umgesetzt und mit verschiedenen alternativen Konzepten im Bereich der Datenbankanbindung verglichen. Die Datenmodellierung setzt die Anforderungen der School of Design Thinking um und verwendet dabei unter anderem Datenversionierung und Datenzugriffsschutz auf Datenmodell-Ebene. Zur Anbindung von externen Komponenten und Systemen wurde ein Eventsystem umgesetzt, welches die Anpassung der Anwendung an zukünftige Bedürfnisse der School of Design Thinking erlaubt. Den Abschluss der Arbeit bildet eine Evaluierung der Anforderungen sowie ein Ausblick auf mögliche Erweiterungen.

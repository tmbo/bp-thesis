%!TEX root = ../Bachelorarbeit.tex
\addchap*{Zusammenfassung}

Im Verlauf der kreativen Design-Prozesse in der HPI School of Design Thinking (D-School) entstehen eine Menge an Dokumentations\gls{Artefakt}en. Diese Daten sind jedoch meist in Quellen abgelegt, deren gesicherte Persistenz nur für einen beschränkten Zeitraum gegeben ist. Die Informationen aus den Wissensquellen gilt es dauerhaft zu sichern, zu organisieren und für Außenstehende verständlich zu präsentieren. Durch die graphische Darstellung des Prozesses der Ideenfindung können sowohl die Teilnehmer eines Design Thinking-Teams, als auch deren Mentoren, wichtige Erkenntnisse für den weiteren Erfolg einer Idee gewinnen. Neben diesem Überblick über die Entwicklungen in einem Projekt, fehlt der D-School eine Gesamtübersicht über alle Projekte. Diese hilft, Abhängigkeiten der Projekte voneinander zu verdeutlichen, und erleichtert die Akquirierung neuer Projekt-Themen. 

In dieser Bachelorarbeit soll das \gls{Backend} der entwickelten Lösung \tete{Project-Zoom} näher erläutert werden. Dabei werden die Architektur des Systems betrachtet und die in diesem Zusammenhang zugrundeliegenden Entscheidungen erläutert. Dazu werden zu Beginn der Arbeit die Anforderungen des Projektpartners aufgezeigt, welche im Verlauf des Entwicklunsprozesses als Entscheidungsgrundlage galten. Das \gls{Backend} der umgesetzten Anwendung basiert auf einer asynchronen Webarchitektur. Es wird das Konzept des data-centric Designs umgesetzt und mit verschiedenen alternativen Konzepten im Bereich der Datenbankanbindung verglichen. Die Datenmodellierung setzt die Anforderungen der HPI School of Design Thinking um und verwendet dabei unter anderem Datenversionierung und Datenzugriffsschutz auf Datenmodell-Ebene. Zur Anbindung von externen Komponenten und Systemen wurde ein \gls{Eventsystem} umgesetzt, welches die Anpassung der Anwendung an zukünftige Bedürfnisse der School of D-School erlaubt. Den Abschluss der Arbeit bilden eine Evaluierung der Anforderungen sowie ein Ausblick auf mögliche Erweiterungen von Project-Zoom.

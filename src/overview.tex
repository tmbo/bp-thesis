%!TEX root = ../Bachelorarbeit.tex
\section{Überblick}

\subsection{Webanwendung vs. native Applikation}
Bevor die Implementierung des Projektes beginnen konnte, galt es zu klären, welche Technologien verwendet werden. Eine der grundlegenden Entscheidungen war, ob eine Webanwendung oder eine native Anwendung entwickelt wird. 

Auf Grund der Anforderung der D-School, dass die Anwendung auch außerhalb der Gebäude der D-School verwendbar sein muss, liegt eine Webanwendung nahe. Hinzu kommt, dass die Projekt-Mitglieder bereits mit dem Umgang von Webseiten und deren Navigation vertraut sind. Für die Erweiterbarkeit stellt dies ebenfalls einen enormen Vorteil dar, da ein zentrales System gewartet werden kann und neue Funktionen einfach eingespielt werden können. Zudem dreht sich das Projekt um das Thema Dokumentation, bei der oft dazu geneigt wird, sie aufzuschieben. Eine Webseite senkt hier die Hemmschwelle und umgeht die Notwendigkeit einer Verteilung und Installation des Programms.

Eine Trennung der Applikation in Frontend und Backend erlaubt eine klare Funktionstrennung. Für Webapplikationen befindet sich das Backend auf Serverseite und das Frontend im Browser auf Clientseite. Die klassischen Aufgaben der beiden Teile sind:

\begin{itemize}
  \item Backend-Aufgaben:
  \begin{itemize}
    \item Sammeln und zur Verfügung stellen von Daten
    \item Kommunikation mit anderen Servern
    \item Validierung eingehender Daten vom Client
    \item Speicherung von Daten
    \item Business-Logik
  \end{itemize}

\item Frontend-Aufgaben:
  \begin{itemize}
    \item Kommunikation mit dem Backend zur Daten-Synchronisation 
\item Visualisierung der Daten
    \item Interaktion mit dem Nutzer
    \item Feedback an den Nutzer
  \end{itemize}
\end{itemize}

%!TEX root = ../Bachelorarbeit.tex
\chapter{Fazit und Ausblick}
In dieser Arbeit wurde der konzeptionelle Aufbau des umgesetzten IT-Systems für die HPI School of Design Thinking näher erläutert. Es wurde die Architektur beginnend beim Datenmodell, über das Konzept des Data-centric Designs, bis zur Anbindung von externen Datenquellen dargelegt. Dabei wurden die Anforderungen des Projektpartners einbezogen und am Ende deren Umsetzung evaluiert.

Während der Evaluierung des Systems wurden von den Testenden verschiedene Verbesserungen an der Nutzeroberfläche vorgeschlagen (vgl. \cite{bp-tomh}). Diese gilt es zu Evaluieren und mit den Anforderungen der D-School abzugleichen.

Während der Umsetzung des Projekts sind verschiedene Ideen aufgekommen das System zu erweitern. Ein Beispiel ist der Mehrbenutzerbetrieb, in welchem mehrere Studenten gleichzeitig an unterschiedlichen Rechnern den Graphen eines Projektes ändern. Dem Backend fällt hier die Aufgabe zu, diese Änderungen zusammenzuführen und die einzelnen Nutzer auf dem neusten Stand zu halten.

Die Administration des Systems ist ein weiterer Punkt, dem bei der zukünftigen Entwicklung des Projektes mehr Aufmerksamkeit zukommen sollte. Für das erste Testen unwesentliche Funktionalitäten, wie das Administrieren der Nutzer und Projekte, müssen für den Produktivbetrieb umgesetzt werden. Dabei muss in Zusammenarbeit mit der D-School ein Konzept des Datenaustausches mit der internen Projektverwaltung entwickelt und anschließend die entsprechenden Funktionalitäten im Backend umgesetzt werden.

Der Erfolg und die Akzeptanz des Projektes als Dokumentationsvisualiserer hängt vorrangig von der zukünftigen Entwicklung der Nutzerschnittstelle ab. Das Konzept des System wurde von den Studenten sehr gut aufgenommen. Das Ziel muss es also sein, durch die Fortführung der Analyse der Nutzeranforderungen, den kreativen Ideenaustausch mit den Mitarbeitern und Studierenden der D-School aufrechtzuerhalten und so das Projekt ...?
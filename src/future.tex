%!TEX root = ../Bachelorarbeit.tex
\chapter{Fazit und Ausblick}
In dieser Arbeit wurde der konzeptionelle Aufbau des umgesetzten IT-Systems für die HPI School of Design Thinking näher erläutert. Es wurde die Architektur, beginnend beim Datenmodell, über das Konzept des data-centric Designs, bis hin zur Anbindung von externen Datenquellen dargelegt. Dabei wurden die Anforderungen des Projektpartners einbezogen und deren Umsetzung am Ende evaluiert.

Während der Evaluierung des Systems unterbreiteten die Testenden verschiedene Verbesserungen an der Nutzeroberfläche (vgl. \cite{bp-tomh}). Diese gilt es zu Analysieren und mit den Anforderungen der D-School abzugleichen.

Im Zuge der Umsetzung des Projektes sind verschiedene Ideen aufgekommen, das System zu erweitern. Ein Beispiel ist der Mehrbenutzerbetrieb, in welchem mehrere Studenten gleichzeitig an unterschiedlichen Rechnern den Graphen eines Projektes ändern können. Dem \gls{Backend} fällt hier die Aufgabe zu, diese Änderungen zusammenzuführen und die einzelnen Nutzer über den neusten Stand zu informieren.

Die Administration des Systems ist ein weiterer Punkt, dem bei der zukünftigen Weiterentwicklung des Projektes mehr Aufmerksamkeit zukommen sollte. Funktionalitäten, welche für das erste Testen unwesentlich waren, wie das Administrieren der Nutzer und Projekte, müssen für den Produktivbetrieb umgesetzt werden. Dabei muss in Zusammenarbeit mit der D-School ein Konzept des Datenaustausches mit der internen Projektverwaltung entwickelt und anschließend die entsprechenden Funktionalitäten im \gls{Backend} umgesetzt werden.

Der Erfolg und die Akzeptanz des Projektes als Dokumentationsvisualiserer hängt vorrangig von der zukünftigen Entwicklung der Nutzerschnittstelle ab. Das Konzept des Systems wurde von den Studenten sehr gut aufgenommen. Das Ziel muss es also sein, durch die Fortführung der Analyse der Nutzeranforderungen, den kreativen Ideenaustausch mit den Mitarbeitern und Studierenden der D-School aufrechtzuerhalten, um dadurch die Dokumentation der D-School mit Project-Zoom weiter zu optimieren.
%!TEX root = ../Bachelorarbeit.tex
\newglossaryentry{Netzwerkdateisystem}{
  name={Netzwerkdateisystem},
  description={Ein Dateisystem, auf dies von Rechnern im selben Netzwerk zugegriffen werden kann, um zum Beispiel Dateien zu lesen}
}

\newglossaryentry{Artefakt}{
  name={Artefakt},
  description={Eine Informationseinheit, welche sich in dieser Arbeit auf die beim Dokumentationsprozess entstehenden, digitalen Informationen bezieht}
}

\newglossaryentry{Ressource}{
  name={Ressource},
  description={Die digitale Repräsentation eines Artefakts. Verschiedene \gls{Thumbnail}s eines Artefakts sind beispielsweise jeweils eine Ressource}
}

\newglossaryentry{Eventsystem}{
  name={Eventsystem},
  description={Ein System, dessen Architektur auf dem Zusammenspiel der Komponenten durch \tete{Events} (Ereignissen) basiert}
}

\newglossaryentry{Konnektor}{
  name={Konnektor},
  description={Komponente, die den Zugriff auf externe Dienste ermöglicht}
}

\newglossaryentry{Thumbnail}{
  name={Thumbnail},
  description={Miniaturbild, welches als Vorschau für ein Bild oder eine Datei dient}
}
 
\newglossaryentry{Aggregation}{
  name={Aggregation},
  description={Bezeichnet das sammeln von Informationen, meist aus verschiedenen Quellen.}
}

\newglossaryentry{Asynchronitaet}{
  name={Asynchronität},
  description={Beschreibt die zeitliche unabhängigkeit von zwei Ereignissen}
}

\newglossaryentry{Business-Logik}{
  name={Business-Logik},
  description={Teil eines Systems, welcher die Logik für die Umsetzung der Problemstellung implementiert}
}

\newglossaryentry{Framework}{
  name={Framework},
  description={In der Softwareentwicklung ein Rahmen, der unterschiedlichste Funktionalitäten zur Verfügung stellt und innerhalb dessen der Anwendungscode ausgeführt wird}
}

\newglossaryentry{Unveraenderbarkeit}{
  name={Unveränderbarkeit},
  description={Bezüglich des Statuses eines Objektes bedeuted dies, dass zur Laufzeit Attributen eines Objektes nach Erstellung dessen keine neuen Werte mehr zugewiesen werden können (vgl. \cite[p.~31]{immutability})}
}

\newglossaryentry{Backend}{
  name={Backend},
  description={Schicht eines Anwendung, welche bei einer Client-Server Applikation den Anwendungscode des Servers bezeichnet}
}

\newglossaryentry{Frontend}{
  name={Frontend},
  description={Schicht eines Anwendung, welche bei einer Client-Server Applikation den Anwendungscode des Client bezeichnet}
}

\newglossarystyle{super3colleft}{%
  \renewenvironment{theglossary}%
    {\tablehead{}\tabletail{}%
     \begin{supertabular}{@{}>{\bfseries}lp{\glsdescwidth}}}%
    {\end{supertabular}}%
  \renewcommand*{\glossaryheader}{}%
  \renewcommand*{\glsgroupheading}[1]{}%
  \renewcommand*{\glossaryentryfield}[5]{%
    \glsentryitem{##1}\glstarget{##1}{##2} & ##3 ##5\\}%
  \renewcommand*{\glossarysubentryfield}[6]{%
     &
     \glssubentryitem{##2}%
     \glstarget{##2}{\strut}##4 ##6\\}%
  \renewcommand*{\glsgroupskip}{ & \\}%
}

\newlength{\acronymlabelwidth}
\setlength{\acronymlabelwidth}{0.25\textwidth}
\newglossarystyle{listwithwidth}{%
  \renewenvironment{theglossary}%
    {\begin{description}}{\end{description}}%
  \renewcommand*{\glossaryheader}{}%
  \renewcommand*{\glsgroupheading}[1]{}%
  \renewcommand*{\glossaryentryfield}[5]{%
    \item[\parbox{\acronymlabelwidth}{\glsentryitem{##1}\glstarget{##1}{##2}}]
       ##3\glspostdescription\space ##5}%
  \renewcommand*{\glossarysubentryfield}[6]{%
    \glssubentryitem{##2}%
    \glstarget{##2}{\strut}##4\glspostdescription\space ##6.}%
  \renewcommand*{\glsgroupskip}{\indexspace}%
}
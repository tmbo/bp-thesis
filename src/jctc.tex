%!TEX root = ../Bachelorarbeit.tex
\chapter{JSON Coast-To-Coast Design}
In diesem Kapitel wird die grundlegende Architektur der Anbindung eines persistenten Speichers im Core von Project-Zoom näher erläutert. Dieser Befindet sich im \tete{Common} Subprojekt im Paket \tete{projectZoom.core} und \tete{models}. Das gewählte Design hat vorrangig Auswirkungen auf die Art und Weise wie der Code für Datenmodelle geschrieben wird, erstreckt sich aber als Betrachtungsweise über die gesamte Architektur.

Das JSON Coast-To-Coast Design ist erstmals zusammenhängend und beispielunterlegt dargestellt wurden durch Pascal Voitet in seinem Blog Mandubian \cite{jctc}. Voitet ist selbst Mitwirkender am open Source Projekt Play und aktiver Scala Bibliotheken Autor (\tete{play-reactivemongo}\footnote{\url{ https://github.com/zenexity/Play-ReactiveMongo}}, \tete{play-autosource}\footnote{\url{ https://github.com/mandubian/play-autosource}}). Für das Design grundlegend sind zwei Entwicklungen. Zum einen das Aufstreben von NoSQL Datenbanken und zum anderen asynchrone Datenbanktreiber.

\section{Motivation}

\subsection{NoSQL}
Die Entwicklung der Datenbank Management Systeme bestand viele Jahre lang in der Optimierung und Verbesserung bestehender relationaler Datenbank Modelle. Im Jahr 1998 kam dann der Term Not-only SQL (NoSQL) auf \cite{storage-solutions}. Heute gibt es verschiedene etablierte Datenbanken die keine reinen SQL Datenbanken mehr sind, bekannte Beispiele sind MongoDB, CouchDB, Apache Cassandra. MongoDB ist eine Open-Source Datenbank, welche ihre Daten Dokumenten basiert speichert.

\subsection{Datenbanktreiber}
Für die Anbindung von relationalen Datenbanken gibt es in Java die Java Database Connectivity (JDBC) Schnittstelle. Diese abstrahiert über Datenbanken und deren Treiber indem eine einheitliche API angeboten wird. Ausgerichtet ist JDBC auf relationale Datenbanken.
Um NoSQL Datenbanken anzubinden benötigt man, wie bei JDBC, einen eigenen Treiber. Der Unterschied ist, das es hier keine Abstraktionsebene über verschiedene NoSQL Datenbanken gibt. Dies liegt vorrangig an der sich stark unterscheidenden Struktur der einzelnen Speichersysteme. Die unterschiedlichen NoSQL Datenbanken sind jeweils speziell auf eine bestimmte Aufgabe ausgerichtet, wie z.B. Durchsatz, Verteilte Umgebungen oder Flexible Daten Schemas. 
\tete{ReactiveMongo} ist ein asynchroner Datenbanktreiber für MongoDB und die Programmiersprache Scala. Die Vorteile eines asynchronen Treibers liegen auf der Hand: Für jede synchrone Datenbank Abfrage wird normalerweise ein Thread verwendet, der bis zur Antwort blockiert ist. Bei mehreren Datenbankabfragen pro Request werden bei Last viele Threads benötigt um Datenbank abfragen auszuführen. Asynchrone Treiber umgehen dieses Problem indem sie Threads, welche Datenbank Abfragen ausführen, nicht blockieren. Für dieses Konzept ist es Notwendig Platzhalter einzuführen. Diese ersetzen das Ergebnis solange es noch nicht vorhanden ist.

\begin{lstlisting}[caption=Funktionssignatur für asynchronen Datenbankzugriff, label=lst:future]
def findOneById(bid: BSONObjectID): Future[Option[Graph]]
\end{lstlisting}
 
Im Beispiel \ref{lst:future} zeigt die Funktionssignatur den Rückgabetyp \tete{Future[Option[Graph]]}. \tete{Future} ist hierbei genau dieser Platzhalter für ein Ergebnis welches noch nicht existiert. Arbeiten kann man mit einem \tete{Future} mit Hilfe von Callbacks.
